\documentclass[10pt]{extarticle}

\usepackage{fullpage}
\usepackage[utf8]{inputenc}
\usepackage[russian]{babel}
\usepackage{graphicx}
\usepackage{booktabs}
\usepackage{amsmath,amsfonts,amssymb}
\usepackage{mathtools}
\usepackage{beamerarticle}
\usepackage{multirow}
\usepackage{indentfirst}
\usepackage{array}
\usepackage{float}
\usepackage{hyperref}

\usepackage{geometry}
\geometry{
	a4paper,
	total={170mm,257mm},
	left=20mm,
	top=20mm,
}

\begin{document}
	\section*{Список вопросов, выносимых на зачёт (2017 год)}
	\begin{enumerate}
		\item Открытые и замкнутые множества на прямой. Канторово множество и его свойства.
		\item Свойства внешней меры. Измеримость открытого множества и счётного объединения открытых множеств. Измеримость замкнутого множества, дополнения измеримого множества, разности и счётного пересечения измеримых множеств.
		\item Свойство счётной аддитивности ($\sigma$-аддитивности) меры. Множества типа $G_{\delta}$ и $F_{\delta}$. Пример неизмеримого множества.
		\item Измеримые функции и их свойства. Измеримость верхнего и нижнего пределов последовательности измеримых функций.
		\item Измеримость предела сходящейся почти всюду последовательности измеримых функций. Сходимость по мере. Связь между сходимостью по мере и сходимостью почти всюду.
		\item Теорема Рисса. Эквивалентность функций, являющихся пределами по мере одной последовательности измеримых функций.
		\item Интеграл Лебега от ограниченной функции. Интегрируемость ограниченной и измеримой функции на множестве конечной меры.
		\item Свойства интеграла Лебега от ограниченной функции.
		\item Интеграл Лебега от неограниченной и неотрицательной функции. Полная аддитивность и абсолютная непрерывность интеграла Лебега. Маорантный признак интегрируемости.
		\item Интеграл Лебега от неограниченной функции любого знака. Теорема Лебега о предельном переходе под знаком интеграла.
		\item Полная аддитивность и абсолютная непрерывность интеграла Лебега от неограниченной функции любого знака. Теорема Леви и её следствие для рядов. Теорема Лебега --- критерий интегрируемости.
		\item Теорема Фубини. Интеграл Лебега для множества бесконечной меры.
		\item Классы $L_p, p > 1$. Неравенства Гёльдера и Минковского.
		\item Полнота пространства $L_p$.
		\item Плотность множества непрерывных функций в $L_p$. Непрерывность в метрике $L_p$.
		\item Метрические пространства. Теорема о вложенных шарах.
		\item Принцип сжимающих отображений. Теорема Бэра о категориях.
		\item Линейные нормированные пространства. Теорема Рисса.
		\item Линейные операторы и их свойства. Теорема о полноте пространства линейных ограниченных операторов.
		\item Теорема Банаха-Штейнгауза (принцип равномерной ограниченности) и следствие из неё. Пример из теории рядов Фурье на применение теоремы Банаха-Штейнгауза.
		\item Обратный оператор. Достаточные условия существования обратного оператора.
		\item Теорема Банаха об обратном операторе.
		\item Теорема Хана-Банаха о продолжении линейного функционала в линейном нормированном пространстве.
		\item Общий вид линейного функционала в конкретных пространствах.
		\item Слабая сходимость. Связь между сильной и слабой сходимостью. Критерий сильной сходимости.
		\item Определение гильбертова пространства и его основные свойства. Теорема об элементе с наименьшей нормой.
		\item Теорема Леви об ортогональной проекции. Разложение гильбертова пространства на прямую сумму подпространства и его ортогонального дополнения.
		\item Теорема Рисса-Фреше об общем виде линейного функционала в гильбертовом пространстве.
		\item Ортонормированные системы. Ортогонализация по Шмидту. Неравенство Бесселя. Полнота и замкнутость ортонормированной системы. Слабая сходимость её к нулю.
		\item Теорема о существовании ортонормированного базиса в сепарабельном гильбертовом пространстве. Теорема об изоморфизме и изометрии всех сепарабельных гильбертовых пространств.
		\item Теорема Рисса-Фишера. Теорема о слабой компактности сепарабельного гильбертова пространства. 
		\item Сопряжённый оператор. Теорема о сопряжённом операторе. Теорема о прямой сумме замыкания образа линейного ограниченного оператора и ядра сопряжённого.
		\item Вполне непрерывный оператор. Пример интегрального вполне непрерывного оператора. Свойства вполне непрерывного оператора.
		\item Первая теорема Фредгольма.
		\item Вторая теорема (альтернатива) Фредгольма.
		\item Третья теорема Фредгольма.
		\item Понятие о спектре линейного оператора в бесконечномерных пространствах. Теорема Гильберта-Шмидта.
	\end{enumerate}
\end{document}
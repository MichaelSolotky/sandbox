\documentclass[]{beamer}

\mode<presentation> {
	%\usetheme{default}
	\usetheme{AnnArbor}
	%\usetheme{Antibes}
	%\usetheme{Bergen}
	%\usetheme{Berkeley}
	%\usetheme{Berlin}
	%\usetheme{Boadilla}
	%\usetheme{CambridgeUS}
	%\usetheme{Copenhagen}
	%\usetheme{Darmstadt}
	%\usetheme{Dresden}
	%\usetheme{Frankfurt}
	%\usetheme{Goettingen}
	%\usetheme{Hannover}
	%\usetheme{Ilmenau}
	%\usetheme{JuanLesPins}
	%\usetheme{Luebeck}
	%\usetheme{Madrid}
	%\usetheme{Malmoe}
	%\usetheme{Marburg}
	%\usetheme{Montpellier}
	%\usetheme{PaloAlto}
	%\usetheme{Pittsburgh}
	%\usetheme{Rochester}
	%\usetheme{Singapore}
	%\usetheme{Szeged}
	%\usetheme{Warsaw}

	%\usecolortheme{albatross}
	\usecolortheme{beaver}
	%\usecolortheme{beetle}
	%\usecolortheme{crane}
	%\usecolortheme{dolphin}
	%\usecolortheme{dove}
	%\usecolortheme{fly}
	%\usecolortheme{lily}
	%\usecolortheme{orchid}
	%\usecolortheme{rose}
	%\usecolortheme{seagull}
	%\usecolortheme{seahorse}
	%\usecolortheme{whale}
	%\usecolortheme{wolverine}
}

\usepackage[utf8]{inputenc}
\usepackage[russian]{babel}
%\usepackage[T2A]{fontenc}
%\usepackage[T1]{fontenc}
%\usepackage{tgtermes}
\usepackage{verbatim}
%\usepackage{minted}

%\usepackage{citehack} 
%\usepackage{tikz}
%\usetikzlibrary{patterns}

%\usepackage{amsmath}
%\usepackage{amsfonts}
%\usepackage{pgfplots}
\usepackage{hyperref}
\usepackage{graphics, graphicx}

%\usepackage{color}
%\usepackage{enumerate}
%\usepackage{booktabs}
%\usepackage{mathtools}
%\usepackage{algorithm}
%\usepackage[noend]{algpseudocode}
%\usepackage{latexsym}
%\usepackage{mdframed}
%\usepackage{minted}
%\usemintedstyle{default}
%\usepackage[noend]{algorithmic}

\title[Git]{\bfseries Обзор системы контроля версий Git}
\author[Cолоткий~М.]{Солоткий Михаил}
\subtitle{Практикум на ЭВМ 2017/2018}
\institute[ВМК МГУ]{МГУ имени М. В. Ломоносова, факультет ВМК, кафедра ММП}
\date{14 ноября 2017 года}

\begin{document}

%\medskip

\begin{frame}
	\titlepage
\end{frame}

\begin{frame} \frametitle{Что такое система контроля версий и зачем она нужна.}
	\begin{itemize}
		\item Система контроля версий (СКВ) — это система, регистрирующая изменения в одном или нескольких файлах с тем, чтобы в дальнейшем была возможность вернуться к определённым старым версиям этих файлов. \newline \par
		\item Вообще, если, пользуясь СКВ, вы всё испортите или потеряете файлы, всё можно будет легко восстановить.
	\end{itemize}
\end{frame}

\begin{frame} \frametitle{Преимущества Git.}
	\begin{itemize}
		\item Быстрое переключение между версиями (в отличие от CVS, Subversion, Perforce, Bazaar); \newline \par
		\item Git - распределённая система (в отличие от CVS, Subversion и Perforce); \newline \par
		\item Git следит за безопасностью данных. \newline \par
	\end{itemize}
\end{frame}

\begin{frame}[fragile] \frametitle{Установка git - утилиты для системы Git.}
	\begin{itemize}
		\item Windows \newline
		Скачиваем .exe инсталятор с \url{http://msysgit.github.com/}
		\item[~]

		\item Mac \newline
		Можно скачать графический инсталятор. \newline
		\url{http://sourceforge.net/projects/git-osx-installer/}
		\item[~]

		\item Linux
		\begin{verbatim}
			$ yum install git-core # Fedora
			$ apt-get install git # Debian or Ubuntu
		\end{verbatim}
		\item[~]
	\end{itemize}
\end{frame}

\begin{frame}[fragile] \frametitle{Как начать контроль версий с git.}
	\begin{itemize}
		\item Создать пустой репозиторий с 0 зафиксированных изменений.
		\begin{verbatim}
			$ git init
		\end{verbatim}
		\item[~]

		\item Скопировать существующий репозиторий со всей историей.
		\begin{verbatim}
			$ git clone https://github.com/esokolov/ml-course-msu.git
		\end{verbatim}
	\end{itemize}
\end{frame}

\begin{frame}[fragile] \frametitle{Коммиты.}
	\begin{itemize}
		\item Добавляем отслеживание файлов.
		\begin{verbatim}
			$ git add . # Для всех файлов
			$ git add README # Для файла README
		\end{verbatim}
		\item[~]
		\item Фиксируем изменение.
		\begin{verbatim}
			$ git commit -m 'first commit'
			[master (root-commit) 11730f0] first commit
			1 file changed, 1 insertion(+)
			create mode 100644 README
		\end{verbatim}
	\end{itemize}
\end{frame}

\begin{frame}[fragile] \frametitle{Ветвление.}
	\begin{itemize}
		\item Ветка - просто указатель на последний коммит.
		\item[~]

		\item Создание ветки iss53 и работа в ней.
		\begin{verbatim}
			$ git checkout -b iss53
			$ vim index.html
			$ git commit -a -m 'added a new footer [issue 53]'
		\end{verbatim}
	\end{itemize}
\end{frame}

\begin{frame}[fragile] \frametitle{Ветвление.}
	\begin{itemize}
		\item Слияние ветки iss53 с веткой master.
		\begin{verbatim}
			$ git checkout master
			$ git merge iss53
		\end{verbatim}

		\begin{center}
			\includegraphics[scale=0.4]{git-merge-normal}
		\end{center}

		\item Удаление ветки iss53.
		\begin{verbatim}
			$ git branch -d iss53
		\end{verbatim}
	\end{itemize}
\end{frame}

\begin{frame}[fragile] \frametitle{Откат до какого-то из прошлых коммитов.}
	\begin{itemize}
		\item Небезопасный способ:
		\begin{verbatim}
			$ git checkout -- README
		\end{verbatim}
		\item[~]

		\item Безопасный способ:
		\begin{verbatim}
			$ git log # Чтобы узнать hash коммита
			$ git checkout <commit hash>
			$ git checkout -b new_branch
		\end{verbatim}
		\item[~]
		\item Больше самых разных способов: \newline
		\href{https://ru.stackoverflow.com/questions/431520/%D0%9A%D0%B0%D0%BA-%D0%B2%D0%B5%D1%80%D0%BD%D1%83%D1%82%D1%8C%D1%81%D1%8F-%D0%BE%D1%82%D0%BA%D0%B0%D1%82%D0%B8%D1%82%D1%8C%D1%81%D1%8F-%D0%BA-%D0%B1%D0%BE%D0%BB%D0%B5%D0%B5-%D1%80%D0%B0%D0%BD%D0%BD%D0%B5%D0%BC%D1%83-%D0%BA%D0%BE%D0%BC%D0%BC%D0%B8%D1%82%D1%83}{https://ru.stackoverflow.com/questions/431520/Как-вернуться-к-более-раннему-коммиту}
	\end{itemize}
\end{frame}

\begin{frame} \frametitle{Самое важное, что нужно знать о git:}
	\begin{itemize}
		\item Как инициировать контроль версий;
		\item Как делать коммиты;
		\item Как переключиться на другую версию;
		\item Как создавать, переключаться, удалять и сливать ветки;
		\item Как работать с удалённым сервером;
		\item Как разрешать конфликты
		\item[~]
	\end{itemize}
	Книга по пользованию git: \newline
	\url{https://git-scm.com/book/ru/v1/}
\end{frame}

\end{document}
\documentclass[10pt]{extarticle}

\usepackage{amsmath,amsfonts,amssymb}
\usepackage{array}
\usepackage{beamerarticle}
\usepackage{booktabs}
\usepackage{filecontents}
\usepackage{float}
\usepackage{fontawesome}
\usepackage{fullpage}
\usepackage{geometry}
\usepackage{graphicx}
\usepackage[hidelinks]{hyperref}
\usepackage{indentfirst}
\usepackage{mathtools}
\usepackage{multirow}
\usepackage[russian]{babel}
\usepackage[utf8]{inputenc}

\begin{filecontents*}{trial.bib}
 @article{paper,
author={name},
title="{title}",
journal={J. Phys. Conf. Ser.},
volume={67},
number={},
 pages={23},
 year={2007}
 }
\end{filecontents*}

\geometry{
    a4paper,
    total={170mm,257mm},
    left=20mm,
    top=20mm,
}

\newcommand{\Expect}{\mathsf{E}}
\newcommand{\MExpect}{\mathsf{M}}
\newcommand{\RomanNumeralCaps}[1]{\MakeUppercase{\romannumeral #1}}

\definecolor{linkcolor}{HTML}{0000FF} % цвет ссылок
\definecolor{urlcolor}{HTML}{0000FF} % цвет гиперссылок
 
\hypersetup{pdfstartview=FitH,  linkcolor=linkcolor,urlcolor=urlcolor, colorlinks=true}

\title{Название}
\author{Солоткий Михаил, 417 группа ВМК МГУ}

\begin{document}
\maketitle
    \section*{Список вопросов, выносимых на зачёт (2017 год)}
    \section{Список вопросов, выносимых на зачёт (2017 год)}

    \cite{paper}

    \bibliographystyle{unsrt}
    \bibliography{trial}

    Consider 2 random variables : c is the true class of the object, k is the number of assessors, who labeled the object as class 1.

    $P(k = m | c = 1) = C_n^m  q^{n - m}  (1 - q)^{m}$, i.e. there are m correct answers and n - m wrong answers
    $P(k = m | c != 1) = C_n^m  q^{m}  (1 - q)^{n - m}$, i.e. there are m wrong and n - m correct answers.
    According to the Bayes theorem: $P(c | k) = P(k | c) P(c)  /  ( \sum_{v \in dom P(c)}  P(k | c = v) P(c = v) )$
    $P(c | k = m) = P(k = m | c) P(c)  /  (C_n^m  q^{n - m}  (1 - q)^{m}  p +  C_n^m  q^{m}  (1 - q)^{n - m}  (1 - p) )$
    $P(c = 1 | k = m) = C_n^m  q^{n - m}  (1 - q)^{m}  p  /  (C_n^m  q^{n - m}  (1 - q)^{m}  p +  C_n^m  q^{m}  (1 - q)^{n - m}  (1 - p) ) = q^{n - m}  (1 - q)^{m} p  /  ( q^{n - m}  (1 - q)^{m} p  +  q^{m}  (1 - q)^{n - m}) (1 - p) )$

    \href{http://www.sharelatex.com}{Something Linky}

\end{document}
%%%%%%%%%%%%%%%%%%%%%%%%%%%%%%%%%%%%%%%%%
% Beamer Presentation
% LaTeX Template
% Version 1.0 (10/11/12)
%
% This template has been downloaded from:
% http://www.LaTeXTemplates.com
%
% License:
% CC BY-NC-SA 3.0 (http://creativecommons.org/licenses/by-nc-sa/3.0/)
%
%%%%%%%%%%%%%%%%%%%%%%%%%%%%%%%%%%%%%%%%%

%----------------------------------------------------------------------------------------
%	PACKAGES AND THEMES
%----------------------------------------------------------------------------------------

\documentclass{beamer}

\mode<presentation> {

% The Beamer class comes with a number of default slide themes
% which change the colors and layouts of slides. Below this is a list
% of all the themes, uncomment each in turn to see what they look like.

%\usetheme{default}
%\usetheme{AnnArbor}
%\usetheme{Antibes}
%\usetheme{Bergen}
%\usetheme{Berkeley}
%\usetheme{Berlin}
%\usetheme{Boadilla}
%\usetheme{CambridgeUS}
%\usetheme{Copenhagen}
%\usetheme{Darmstadt}
%\usetheme{Dresden}
%\usetheme{Frankfurt}
%\usetheme{Goettingen}
%\usetheme{Hannover}
%\usetheme{Ilmenau}
%\usetheme{JuanLesPins}
%\usetheme{Luebeck}
\usetheme{Madrid}
%\usetheme{Malmoe}
%\usetheme{Marburg}
%\usetheme{Montpellier}
%\usetheme{PaloAlto}
%\usetheme{Pittsburgh}
%\usetheme{Rochester}
%\usetheme{Singapore}
%\usetheme{Szeged}
%\usetheme{Warsaw}

% As well as themes, the Beamer class has a number of color themes
% for any slide theme. Uncomment each of these in turn to see how it
% changes the colors of your current slide theme.

%\usecolortheme{albatross}
%\usecolortheme{beaver}
%\usecolortheme{beetle}
%\usecolortheme{crane}
%\usecolortheme{dolphin}
%\usecolortheme{dove}
%\usecolortheme{fly}
%\usecolortheme{lily}
%\usecolortheme{orchid}
%\usecolortheme{rose}
%\usecolortheme{seagull}
%\usecolortheme{seahorse}
%\usecolortheme{whale}
%\usecolortheme{wolverine}

%\setbeamertemplate{footline} % To remove the footer line in all slides uncomment this line
%\setbeamertemplate{footline}[page number] % To replace the footer line in all slides with a simple slide count uncomment this line

%\setbeamertemplate{navigation symbols}{} % To remove the navigation symbols from the bottom of all slides uncomment this line

}

\usepackage[utf8]{inputenc}
\usepackage[russian]{babel}
\usepackage{graphicx} % Allows including images
\usepackage{booktabs} % Allows the use of \toprule, \midrule and \bottomrule in tables
\usepackage{xcolor}
\usepackage[export]{adjustbox}
\usepackage{caption}
\usepackage{subcaption}
\setbeamertemplate{theorems}[numbered]
\setbeamertemplate{corollary}[numbered]
\setbeamertemplate{definitions}[numbered]
\deftranslation[to=russian]{Theorem}{Теорема}
\deftranslation[to=russian]{Corollary}{Следствие}
\deftranslation[to=russian]{Definition}{Определение}
%\newtheorem{theorem}{Theorem}

%----------------------------------------------------------------------------------------
%	TITLE PAGE
%----------------------------------------------------------------------------------------

\title[]{Алгоритмы анализа поведения реагирующих систем реального времени}

\author{Винарский Е. М.}
\institute[ВМК МГУ] % Your institution as it will appear on the bottom of every slide, may be shorthand to save space
{
\medskip
\textit{Московский государственный университет им. М.В.Ломоносова
\newline
Факультет вычислительной математики и кибернетики
\newline
Кафедра математической кибернетики
} \\

Научный руководитель: Захаров Владимир Анатольевич
}
\date{\today} % Date, can be changed to a custom date

\begin{document}

\begin{frame}
\titlepage % Print the title page as the first slide
\end{frame}

%\begin{frame}
%\frametitle{Содержание}
%\tableofcontents % Throughout your presentation, if you choose to use \section{} and \subsection{} commands, these will automatically be printed on this slide as an overview of your presentation
%\end{frame}

%----------------------------------------------------------------------------------------
%	PRESENTATION SLIDES
%----------------------------------------------------------------------------------------

%------------------------------------------------

%------------------------------------------------
%------------------------------------------------

%new presentation

%------------------------------------------------
\section{Введение}

%------------------------------------------------
\begin{frame}
\frametitle{Модель <<реального>> времени}

% картинка

Для моделирования систем <<реального>> времени простейшей моделью является временной автомат-преобразователь (трансдьюсер)

В отличие от классического трансдьюсера:
\begin{itemize}
    \item поведение временного трансдьюсера зависит не только от порядка следования входных символов, но и от их времени поступления
    
    \item выходные символы выдаются по прошествии времени, называемого выходной задержкой
\end{itemize}

Необходимо учесть это обстоятельство при изучении поведения таких систем

\end{frame}

%------------------------------------------------

%------------------------------------------------
\section{Обзор известных результатов}

%------------------------------------------------
\begin{frame}
\frametitle{Обзор известных результатов для временных автоматов}

Самой общей является модель временного автомата, которая включает:
\begin{enumerate}
    \item несколько независимых временных переменных
    \item временные ограничения
    \item переходы по таймаутам
    \item возможность сброса времени на переходах
\end{enumerate}
 
Класс языков, распознаваемых такими временными автоматами, не замкнут относительно операции дополнения

Проблемы тотальности и проблема проверки эквивалентности для таких временных автоматов алгоритмически неразрешимы

Модель временного автомата-преобразователя с одной временной переменной -- компромиссный вариант между сложностью и адекватностью модели

\end{frame}

%------------------------------------------------
\begin{frame}
\frametitle{Цель Выпускной квалификационной работы}

Целью ВКР являлось:
\begin{enumerate}
    \item Исследовать свойства моделей временного автомата и временного трансдьюсера
    
    \item Предложить алгоритм проверки свойств для таких моделей <<реального>> времени
\end{enumerate}

%Мы предлагаем:
%\begin{enumerate}
%	\item Для описания допустимых последовательностей сигналов использовать временные автоматные языки
	
%	\item В качестве реагирующих систем <<реального>> времени использовать временные трансдьюсеры
	
%	\item Для описания свойств поведения временных трансдьюсеров использовать $\mathcal{LP}$-RLTL логику, которая является расширением линейной темпоральной логики LTL
%\end{enumerate}

\end{frame}

%------------------------------------------------
\begin{frame}
\frametitle{Сквозной пример}



Вопрос: существует ли входное временное слово, допускаемое $\mathcal{A}(p_0)$ такое, что выходное временное слово допускается $\mathcal{B}(q_0)$?

%\pause

\begin{figure}
\centering
\begin{subfigure}{.5\textwidth}
  \centering

  \caption{Временной автомат $\mathcal{A}(p_0)$}
\end{subfigure}%
\begin{subfigure}{.5\textwidth}
  \centering
  \caption{Временной автомат $\mathcal{B}(q_0)$}
\end{subfigure}
\end{figure}

\end{frame}

%------------------------------------------------
\begin{frame}
\frametitle{Постановка задачи}

В рамках ВКР предполагалось решить следующие задачи:
\begin{enumerate}
	\item Ввести достаточно выразительный класс временных автоматных языков, который замкнут относительно теоретико-множественных операций объединения, пересечения, итерации и дополнения
  	
  	\item Разработать расширение LTL логики, в которой можно было бы описывать свойства <<реального>> времени
  	
  	\item Разработать алгоритм проверки выполнимости формулы для такой логики %$\mathcal{LP}$-RLTL логики для случая, когда временной трансдьюсер обладает свойством консервативности, то есть выходные символы располагаются в выходном слове в том же порядке, что и соответствующие им входные символы 
%  	\item Оценить сложность этого алгоритма.
\end{enumerate}

\end{frame}

%------------------------------------------------
\section{Временные слова}
%------------------------------------------------

\begin{frame}
\frametitle{Временные слова}
% временные слова вводятся обычным образом

Последовательность пар $ (\sigma_1, t_1), (\sigma_2, t_2), \dots, (\sigma_n, t_n)$ -- \emph{входное (выходное) временным словом}, где:

\begin{itemize}
	\item $\Sigma$ -- конечный \emph{(входной или выходной) алфавит}
    \item $\sigma_j \in \Sigma$
	\item $t_j \in \mathbf{R}^{+}_0$ -- временная метка для $\sigma_j$
	\item $t_1, t_2, \dots, t_n$ -- возрастающая последовательность
\end{itemize}

\end{frame}

%------------------------------------------------

% определяем сразу временной трансдьюсер

%------------------------------------------------
\section{Временной автомат}
%------------------------------------------------

% ввести понятие временного слова, как обычно. Объект, работающий с временными словами -- временной автомат.

\begin{frame}
\frametitle{Временной автомат}
% определяется обычным образом

\emph{Временным автоматом} над алфавитом входных символов $\Sigma$ называется кортеж $\mathcal{M} = (S, G, \rho, F)$
	\begin{itemize}
		\item $S$ -- конечное множество состояний
		\item $G$ -- множество допустимых интервалов, то есть интервалов вида $[u, v)$, где $u, v \in \mathcal{R^{+}}$ и $u \leq v$
		\item $\rho \subseteq S \times \Sigma \times G \times S$ -- отношение переходов
		\item $F \subseteq S$ -- множество финальных состояний
	\end{itemize}

Временной автомат $\mathcal{M}$ -- \emph{инициальный}, если выделено начальное состояние $s \in S$; обозначим $\mathcal{M}(s)$
	
Временной автомат называется \emph{детерминированным}, если для любых двух переходов вида $(s, \sigma, g_1, s')$ и $(s, \sigma, g_2, s'')$ выполняется $g_1 \cap g_2 = \emptyset$

\end{frame}

%------------------------------------------------

\begin{frame}
\frametitle{Временной автомат (2)}

% пример в конец

\begin{itemize}
    \item $(\sigma_1, 5), (\sigma_2, 8)$ -- \emph{недопустимое временное слово}
    \item $(\sigma_1, 3), (\sigma_2, 5)$ -- \emph{допустимое временное слово}, но не принимаемое временным автоматом
    \item $(\sigma_1, 3), (\sigma_2, 5), (\sigma_3, 8),$ -- \emph{допустимое временное слово}, \emph{принимаемое} временным автоматом
\end{itemize}

\textit{Язык временного автомата} -- все допустимые слова, принимаемые временным автоматом

Временные автоматы $\mathcal{M}_1(s_0)$ и $\mathcal{M}_2(s_0)$ \textit{эквивалентны}, если $L(\mathcal{M}_1(s_0)) = L(\mathcal{M}_2(s_0))$

\end{frame}
%------------------------------------------------

%------------------------------------------------
\section{Замкнутость временных автоматных языков относительно теоретико-множественных операций}
%------------------------------------------------ 

\begin{frame}
\frametitle{Замкнутость временных автоматных языков относительно теоретико-множественных операций}

В рамках ВКР доказаны следующие теоремы:
\begin{theorem}
{\itshape
Для любого \textcolor{red}{недетерминированного} полностью определённого временного автомата $\mathcal{M}(s_0)$ над алфавитом $\Sigma$ существует временной \textcolor{red}{детерминированный} полностью определённый автомат $\mathcal{M'}(s_0)$ над алфавитом $\Sigma$, такой, что $L(\mathcal{M}(s_0)) = L(\mathcal{M'}(s_0)))$
} 
\end{theorem}

\begin{theorem}
{\itshape
Класс временных автоматных языков замкнут относительно теоретико-множественных операций объединения, пересечения, конкатенации, итерации и дополнения
}
\end{theorem}

\end{frame}

%------------------------------------------------
\section{Временной трансдьюсер}
%------------------------------------------------

\begin{frame}
\frametitle{Временной трансдьюсер}

\emph{Временным трансдьюсером} над алфавитами $\Sigma$ и $\Delta$ называется кортеж $\mathcal{M} = (S, s_0, G, D, \rho)$, где:
	\begin{itemize}
		\item $S$ -- конечное непустое \emph{множество состояний}
		\item $s_0 \in S$ -- \emph{начальное состояние}
		\item $G$ -- множество \emph{входных допустимых интервалов}, то есть интервалов вида $(u, v]$, где $u, v \in \mathcal{R^{+}}$ и $0 < u < v$
		\item $D$ -- множество \emph{выходных допустимых интервалов}, то есть интервалов вида $[p, q]$, где $p, q \in \mathcal{R^{+}}$ и $0 \leq p \leq q$
		\item $\rho \subseteq S \times \Sigma \times \Delta \times S \times G \times D$ -- \emph{отношение переходов}
	\end{itemize}

\end{frame}

%------------------------------------------------

\begin{frame}
\frametitle{Временной трансдьюсер (2)}

$(\sigma_1, 1.0), (\sigma_2, 2.7), (\sigma_3, 3.5)$ -- входное временное слово

\textbf{Если} \textcolor{red}{$d_1 = 2$, $d_2 = 2.5$, $d_3 = 1.5$}, \\
\textbf{то} \textcolor{blue}{$(b_1, 3.0), (b_3, 5.2), (b_2, 5.0)$} -- выходное временное слово \\
и \textcolor{blue}{$b_1, b_3, b_2$} -- выходное слово.

%\pause

\textbf{Если} \textcolor{red}{$d_1 = 2$, $d_2 = 2.5$, $d_3 = 2$}, \\
\textbf{то} \textcolor{blue}{$\beta_1 = (b_1, 3.0), (b_2, 5.2), (b_3, 5.5)$} -- выходное временное слово \\
и \textcolor{blue}{$b_1, b_2, b_3$} -- выходное слово.



%\pause

Временной трансдьюсер называется \emph{консервативным}, если для любого допустимого входного временного слова выходные символы располагаются в том же порядке, что и соответствующие им входные символы

\end{frame}

%------------------------------------------------

\section{$\mathcal{LP}-RLTL$ логика}
%------------------------------------------------

\subsection{Автоматные формулы}

\begin{frame}
\frametitle{Автоматные формулы}

\begin{itemize}
    \item $B(s)$ -- элементарная автоматная формула
 
    \item \textbf{Если} $\varphi_1$ и $\varphi_2$ -- автоматные формулы, \textbf{то} $\varphi_1 \wedge \varphi_2$ и $\varphi_1 \vee \varphi_2$ -- автоматные формулы
 
    \item \textbf{Если} $a \in \Sigma$ и $\phi$ -- автоматная формула, \textbf{то} $\mathbf{X}_{(a, t)}\varphi$ -- автоматная формула
 
    \item \textbf{Если} $a \in \Sigma$ и $\varphi$ -- автоматная формула и $\langle u, v \rangle$ -- допустимый интервал, 
    \textbf{то} $\mathbf{X}_{(a, \langle u, v \rangle)}\varphi$ -- автоматная формула
 
    \item \textbf{Если} $\varphi$ -- автоматная формула, \textbf{то} $\mathbf{F}_{A(s)}\varphi$ -- автоматная формула
\end{itemize}

\end{frame}

%------------------------------------------------

\begin{frame}
\frametitle{Определение трассы в $\mathcal{LP}-RLTL$ логика}

Под \emph{трассой} временного трансдьюсера будем понимать четвёрку следующего вида $(\alpha, \beta, \gamma, T)$, где:
\begin{itemize}
	\item $\alpha = (a_1, t_1), (a_2, t_2), \dots, (a_n, t_n)$ -- входное временное слово, $t_{\alpha} = \sum\limits_{j = 1}^{n}t_j$
    \item $\beta = (b_1, \tau_1), (b_2, \tau_2), \dots, (b_n, \tau_n)$ -- выходное временное слово, $\tau_{\beta} = \sum\limits_{j = 1}^{n}\tau_j$
    \item $\gamma = (b_1, \tau_1), (b_2, \tau_2), \dots, (b_m, \tau_m), m \leq n$ -- выходное временное слово, выданное к моменту времени $\tau_{\gamma}$ (контекст)
    \item $T$ -- разность времени между последним поступившим и последним выданным символами (время запаздывания)
\end{itemize}

\end{frame}

%------------------------------------------------
\begin{frame}{Выполнимость формулы на трассе в $\mathcal{LP}-RLTL$ логике}

Определим выполнимость формулы $\varphi$ на трассе 
$(\alpha, \beta, \gamma, T)$

\begin{columns}[t,onlytextwidth]
\column{.5\textwidth}

\begin{itemize}
 \item $(\alpha, \beta, \gamma, T) \models \mathcal{B}(s)$ $\Longleftrightarrow$ $\gamma \in L(\mathcal{B}(s))$
 
 \item $(\alpha, \beta, \gamma, T) \models \mathbf{X}_{(a, t)}\varphi$ 
    $\Longleftrightarrow$
    \begin{equation*}
        \begin{cases}
        
        \alpha = (a, t)\alpha_2 \\
       
        \beta = \beta_1\beta_2, \text{ где }
        \beta_2 = (b, \tau)\beta_2' \\
        
        T' = T + t - \tau_{\beta_1} \geq 0 \\
        
        T' - \tau < 0 \\
        
        (\alpha_2, \beta_2, \gamma\beta_1, T') \models \varphi
	
        \end{cases}
    \end{equation*}
\end{itemize}

\column{.5\textwidth}

\begin{itemize}
    \item $(\alpha, \beta, \gamma, T) \models \mathbf{F}_{A(p)}\varphi$ 
          $\Longleftrightarrow$
    \begin{equation*}
        \begin{cases}

        \exists \alpha_1: (\alpha_1 \in \mathcal{A}(p)) \text{ и } (\alpha = \alpha_1\alpha_2) \\

        \beta = \beta_1\beta_2, \text{ где } 
        \beta_2 = (b, \tau)\beta_2' \\
        
        T' = T + t_{\alpha_1} - \tau_{\beta_1} \geq 0 \\
        
        T' - \tau < 0 \\
    
        (\alpha_2, \beta_2, \gamma\beta_1, T') \models \varphi

        \end{cases}
    \end{equation*}
\end{itemize}

\end{columns}


\end{frame}

%------------------------------------------------

\subsection{Тождество неподвижной точки}

%------------------------------------------------
\begin{frame}
\frametitle{Тождество неподвижной точки}

В рамках ВКР доказана следующая теорема:
\begin{theorem}

Пусть $\mathcal{A} = (S, G, \rho, F)$ -- временной автомат над алфавитом $\Sigma$, тогда в $\mathcal{LP}-RLTL$ логике справедливо тождество

\begin{equation*}
\mathbf{F}_{A(p)}\varphi = 
 \begin{cases}
   \text{$\varphi \vee \bigvee\limits_{(a, (u, v]) \in \widetilde{\Sigma}: p' = \lambda(p, (a, (u, v]))} \mathbf{X}_{(a, (u, v])}\mathbf{F}_{A(p')}\varphi $} & \text{$ p \in F $} \\
   
   \text{$\bigvee\limits_{(a, (u, v]) \in \widetilde{\Sigma}: s' = \lambda(s, (a, (u, v]))} \mathbf{X}_{(a, (u, v])}\mathbf{F}_{A(s')}\varphi $} & \text{$ p \not\in F $}
 \end{cases}
\end{equation*}

\end{theorem}

\end{frame}

%------------------------------------------------
\section{Размеченная система переходов}
%------------------------------------------------
 
\subsection{Размеченная система переходов}

%------------------------------------------------
\begin{frame}
\frametitle{Пример размеченной системы переходов (фрагмент)}

\begin{figure}
\centering
\begin{subfigure}{.5\textwidth}
  \centering
  \caption{Временной автомат параметризации $A(p_0)$}
\end{subfigure}%
\begin{subfigure}{.5\textwidth}
  \centering
  \caption{Временной автомат спецификации $B(q_0)$}
\end{subfigure}
\end{figure}



\end{frame}

%------------------------------------------------
\begin{frame}
\frametitle{Пример размеченной системы переходов (фрагмент) (2)}

Свойство: существует ли входное временное слово, допускаемое $A(p_0)$ такое, что выходное временное слово допускается $B(q_0)$ 

в $\mathcal{LP}-RLTL$ логике выражается следующим образом:

$\varphi = \mathbf{F}_{A(p_0)}B(q_0)$



\end{frame}

%------------------------------------------------
\subsection{Алгоритм проверки модели (Model Checking)}
%------------------------------------------------

\begin{frame}
\frametitle{Алгоритм проверки модели (Model Checking)}

В рамках ВКР доказана теорема: 
\begin{theorem}
{\itshape
Формула $\varphi$ выполнима для консервативного трансдьюсера $\textbf{M}$ если и только если в размеченной системе переходов достижимо финальное состояние
}
\end{theorem}

\end{frame}

%------------------------------------------------

%------------------------------------------------

\section{Полученные результаты}

\begin{frame}
\frametitle{Полученные результаты}

% оставить слайд о замкнутости операций

В работе получены следующие основные результаты:
\begin{enumerate}
	\item Введён класс временных автоматных языков, который замкнут относительно операций объединения, пересечения, итерации и дополнения
  	\item Разработана $\mathcal{LP}$-RLTL логика
  	\item Разработан алгоритм проверки выполнимости формулы $\mathcal{LP}$-RLTL логики для случая, когда временной трансдьюсер обладает свойством консервативности
%  	\item Оценить сложность этого алгоритма.
\end{enumerate}

\end{frame}
%------------------------------------------------

%------------------------------------------------
\section{Приложение}
%------------------------------------------------

\subsection{Унифицированная форма временного автомата}
%------------------------------------------------

\begin{frame}
\frametitle{Унифицированная форма временного автомата}

\begin{figure}
\centering
\begin{subfigure}{.5\textwidth}
  \centering
  \caption{Временной автомат $M$}
\end{subfigure}%
\begin{subfigure}{.5\textwidth}
  \centering
  \caption{Временной автомат $M$ в унифицированной форме}
\end{subfigure}
\end{figure}

$\widetilde{\Sigma} = (\sigma, (1, 2]), (\sigma, (2, 3]), (\sigma, (3, 4]), (\sigma, (6, 9])$ -- \textit{унифицированные временные символы}

\end{frame}
 
%------------------------------------------------
\subsection{Конечно-автоматная абстракция временного автомата}
%------------------------------------------------

\begin{frame}
\frametitle{Конечно-автоматная абстракция}



\begin{theorem}
{\itshape
Язык, распознаваемый временным автоматом, не изменится, если переход $(s, \sigma, (u, v], s')$ заменить на два перехода $(s, \sigma, [u, d), s')$ и $(s, \sigma, [d, v), s')$
}
\end{theorem}

\end{frame}

%------------------------------------------------
\begin{frame}
\frametitle{Конечно-автоматная абстракция (2)}

\begin{itemize}
	\item $\Sigma = \{ \sigma_1, \dots, \sigma_n \}$ -- конечный алфавит;
	\item $M_1 = (S_1, s^1_0, G_1, \rho_1, F_1)$ и $M_2 = (S_2, s^2_0, G_2, \rho_2, F_2)$ -- временные автоматы
	\item $\widetilde{M_1} = (S_1, s^1_0, \widetilde{\rho_1}, F_1)$ и $\widetilde{M_2} = (S_2, s^2_0, \widetilde{\rho_2}, F_2)$ -- конечно-автоматные абстракции временных автоматов $M_1$ и $M_2$
\end{itemize}

Тогда справедлива следующая теорема
\begin{theorem}
{\itshape
$M_1 \sim M_2$ тогда и только тогда, когда $\widetilde{M_1} \sim \widetilde{M_2}$
}
\end{theorem}

\end{frame}
%------------------------------------------------

\begin{frame}
\frametitle{Построение размеченной системы переходов}

\textbf{Идея:} каждое состояние размеченной системы переходов хранит информацию о
\begin{itemize}
	\item текущем состоянии проверяемого трансдьюсера
	\item системе подформул, проверяемых на текущем шаге
	\item конечной последовательности выходных временных символов, которые ещё не успели обработаться временным трансдьюсером
	\item ''накопленном'' времени
	\item финальными состояниями системы переходов будут те состояния, в которых:
    \begin{itemize}
        \item входное временное слово принадлежит языку спецификации
        \item выходное временное слово принадлежит языку контекста
    \end{itemize}
\end{itemize}

\end{frame}

%------------------------------------------------

%\section{План дальнейших работ}

%\begin{frame}
%\frametitle{План дальнейших работ}

%Задачей дальнейших исследований являются:
%\begin{itemize}
%	\item Разработка алгоритма верификации для временных трансдьюсеров общего вида.
%	\item Расширение $\mathcal{LP}$-RLTL логики другими темпоральными операторами.
%\end{itemize}

%\end{frame}

%------------------------------------------------

\end{document}
\grid

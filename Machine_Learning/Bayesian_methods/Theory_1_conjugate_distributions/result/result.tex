\documentclass[10pt]{extarticle}

\usepackage{fullpage}
\usepackage[utf8]{inputenc}
\usepackage[russian]{babel}
\usepackage{graphicx}
\usepackage{booktabs}
\usepackage{amsmath,amsfonts,amssymb}
\usepackage{mathtools}
\usepackage{beamerarticle}
\usepackage{multirow}
\usepackage{indentfirst}
\usepackage{array}
\usepackage{float}
\usepackage{hyperref}

\usepackage{geometry}
\geometry{
	a4paper,
	total={170mm,257mm},
	left=20mm,
	top=20mm,
}

\newcommand{\Expect}{\mathsf{E}}
\newcommand{\MExpect}{\mathsf{M}}

\title{Теоретическое домашнее задание по теме <<Сопряжённые распределения и экспененциальный класс распределений>>}
\author{Солоткий Михаил, 417 группа ВМК МГУ}

\begin{document}
\maketitle
	\section{Сопряжённое к равномерному}
	\subsection{Оценка максимального правдоподобия $\theta_{ML}$}
	Пусть дана простая выборка $x_1, \dots, x_n$  из непрервыного равномерного распределения: \\
	$p(x_i | \theta) = \cfrac{1}{\theta} \cdot [0 \leq x_i \leq \theta]$. Вывести $\theta_{ML}$. \\
	$L(X | \theta) = \prod\limits_{i = 1}^n \cfrac{1}{\theta} \cdot [0 \leq x_i \leq \theta]$ \\
	$L(X | \theta) = 0$ если $\max\limits_{i = 1, \dots, n}(x_i) > \theta$ или $\min\limits_{i = 1, \dots, n}(x_i) < 0$. В последнем случае любое $\theta = \theta_{ML}$ $\forall$ $\theta > 0$ \\
	Предположим все $x_i \geq 0$. Тогда $L(X | \theta) < L(X | \max\limits_{i = 1, \dots, n} x_i)$ $\forall$ $\theta > \max\limits_{i = 1, \dots, n} x_i$. Значит $\theta_{ML} = \max\limits_{i = 1, \dots, n} x_i$.
	\subsection{Сопряжённое априорное и апастериорное распределение}
	$p(\theta | a, b) = \cfrac{b \cdot a^b}{\theta^{b + 1}} \cdot [\theta \geq a]$ \\
	$p(x | \theta) = \cfrac{1}{\theta} \cdot [\theta \geq x]$ \\
	$p(\theta | X, a, b) = \cfrac{p(X | \theta) \cdot p(\theta | a, b)}{\int p(x | \theta) \cdot p(\theta | a, b) d \theta}=$
	$\cfrac{1}{z} \cdot \cfrac{b \cdot a^b}{\theta^{b +n + 1}} \cdot [\theta \geq \max\limits_{i = 1, \dots, n}(a, x_i)]$ \\
	$z = \int\limits_0^{\infty} \cfrac{b \cdot a^b}{\theta^{b +n + 1}} \cdot [\theta \geq \max\limits_{i = 1, \dots, n}(a, x_i)] d \theta =$
	$\int\limits_{\max\limits_{i = 1, \dots, n}(a, x_i)}^{\infty} \cfrac{b \cdot a^b}{\theta^{b +n + 1}} d \theta =$
	$- \cfrac{b \cdot a^b}{\theta^{b + n}} \Bigg|_{\max\limits_{i = 1, \dots, n}(a, x_i)}^{\infty}=$
	$\cfrac{b \cdot a^b}{\Big( \max\limits_{i = 1, \dots, n} (a, x_i) \Big)^{b + n} \cdot (b + n)}$ \\
	$p(\theta | X, a, b) = \cfrac{\Big( \max\limits_{i = 1, \dots, n} (a, x_i) \Big)^{b + n} \cdot (b + n)}{\theta^{b + n + 1}}=$
	$\mathrm{Pareto}(\theta | a', b')$;
	$a' = \max\limits_{i = 1, \dots, n} (a, x_i)$;
	$b' = b + n$.
	\subsection{Вычисление статистик}
	$\xi \sim \mathrm{Pareto}(x | a, b)$ \\
    \begin{itemize}
        \item $\Expect{\xi} =$
	    $\int\limits_a^{\infty} \cfrac{b \cdot a^b}{x^{b + 1}} \cdot x dx =$
	    $\int\limits_a^{\infty} \cfrac{b \cdot a^b}{x^b} dx =$
	    $\cfrac{b \cdot a^b}{(b - 1) \cdot a^{b - 1}} =$
	    $\cfrac{ab}{b - 1}$
	    \item $\cfrac{1}{2} =$
	    $\int\limits_a^c \cfrac{b \cdot a^b}{x^{b + 1}} dx =$
	    $a^b \cdot \Bigg( \cfrac{1}{a^b} - \cfrac{1}{c^b} \Bigg) =$
	    $1 - \Bigg( \cfrac{a}{c} \Bigg)^b$ \\
	    $\Bigg( \cfrac{a}{c} \Bigg)^b = \cfrac{1}{2}$ \\
	    $Med \xi = c = a \cdot 2^{\frac{1}{b}}$
	    \item $Mod \xi = a$
    \end{itemize}
    \section{Задача про автобусы}
    В новом городе я не знаю, какие автобусы ходят чаще каких. Поэтому предположим, что наблюдать любой номер автобуса равновероятно. Предположим, номера автобусов нумеруются подряд, начиная с первого. Обобщим дискретное равномерное распределение на непрерывное таким образом: номер автобуса - целая часть сверху от случайной величины, распределённой непрерывно на $[0, \theta]$. Параметр $\theta$ надо оценить в задаче. Понятно, что миллиарда автобусных маршрутов в городе нет, но чтобы не рассматривать ограниченные множества непонятного размера, загрубим модель: скажем, что возможно любое вещественное $\theta > 0$. Возьмём в качестве априорного распределение Парето:
    $$p(\theta | a, b) = \cfrac{b \cdot a^b}{x^{b + 1}} \cdot [\theta \geq a]$$
    Параметр $a$ можно выбрать исходя из знания от том, сколько автобусов в родном городе. Параметр $b$ отвечает за степень доверия априорным знаниям: чем оно выше, тем больше разница менее вероятны значения, далёкие от $a$. Пронаблюдав автобус с номером 100, а затем автобус с номером 150 параметр a может измениться: если он раньше был больше или равен 150, то не изменится, в противном случае станет равным 150, и значения $\theta$, большие 150 приобретут большую вероятность. При наблюдении автобуса с номером 50, большИе значения параметра $a$ получат меньшую вероятность, так как $b$ увеличится на единицу и хвосты станут легче. \\
    Брать моду распределения в качестве точечной оценки значит не использовать априорной информации (если, конечно в родном горое автобусов было не больше 150). \\
    Брать медиану можно в принципе ровно как и мат. ожидание. В любом из вариантов используется набюдаемая информация и опыт, результат будет больше 150. Насколько больше - определяется параметром b. Мат. ожидание - взвешивание с вероятностями всех допустимых значений, то есть как бы довольно много информации из распределения было взято из апастериорного распределения. В общем мат. ожидание брать адекватно. Никаких за или против медианы у меня нет, слишком неочевидно.
    \section{Экспоненциальный класс: распределение Парето}
    $p(x | \theta) = \cfrac{f(x)}{g(\theta)} \cdot e^{\theta^T u(x)}$\\
    $p(x | a, b) = \cfrac{b \cdot a^b}{x^{b + 1}} \cdot [x \geq a] =$
    $b \cdot a^b \cdot e^{-(b + 1) \cdot \ln x} \cdot [x \geq a] =$
    $\cfrac{b \cdot a^b}{x} \cdot e^{-b \cdot \ln x} \cdot [x \geq a]$ \\
    $g(b) = \cfrac{1}{b \cdot a^b}$;
    $f(x) = \cfrac{[x \geq a]}{x}$ \\
    $\Expect(\ln x) = g(b)' =$
    $-\cfrac{a^b + b \cdot a^b \cdot \ln a}{b^2 \cdot a^{2 \cdot b}} = $
    $-\cfrac{1 + b \cdot \ln a}{b^2 \cdot a^{b}}$
\end{document}